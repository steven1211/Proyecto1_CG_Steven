\documentclass[12pt,letter paper]{article}
\usepackage[utf8]{inputenc}
\usepackage{blindtext}
\usepackage[T1]{fontenc}
\usepackage[spanish]{babel}
\usepackage{graphicx}
\usepackage{subcaption}
\usepackage{listings}
\usepackage{amsmath}
\usepackage{amsfonts}
\usepackage{amssymb}
\usepackage{graphicx}
\usepackage{enumerate}
\usepackage[section]{placeins}

\title{Tarea Corta 2: Historia de los gráficos por computadora en la década de 1990's}
\author{Steven Josué Retana Cedeño\\
\texttt{2017144537}
}
\date{17 de junio de 2021}

\begin{document}

\maketitle
\newpage

\section{Historia de los gráficos por computadora en la década de 1990}

La década de los años 90 para los gráficos por computadora marcaron grandes avances y creaciones en esta tecnología, entre los más destacabables se encuentran la creación de ``Renderman" como parte de Pixar el cual es un motor de renderizado que transformó la industria del cine animado por ejemplo con grandes películas como ``Toy Story'' o ``Los increíbles'' por mencionar sólo algunas.\vspace{\baselineskip}

Además de esto también se desarrolló el VGA (Video Graphics Array) por parte de IBM como mejora a la resolución y la cual tuvo una aceptación masiva en el mercado y otros elementos importantes en la historia como la creación del primer navegador web gráfico.\vspace{\baselineskip}

A continuación se muestran los avances a lo largo de los años 90 con lo más destacable de cada año, dichos avances se pueden apreciar en forma de ``timelapse'' en \cite{Historia} la cual muestra la historia de los gráficos por computadora y los puntos clave de cada año.\vspace{\baselineskip}

\subsection{(1990) - Creación del VGA y el software Renderman }

Este año se destaca por la creación de la VGA (Video Graphics Array) la cual hace referencia al sistema de colores creado por IBM para sus computadoras y procesadores, como se puede ver en \cite{VGA}, esta se convirtió rapidamente en una de las plataformas de color más populares y tuvo una aceptación masiva en el mercado. El soporte VGA (con una resolución de 640 x 480 pixeles) permitía obtener una paleta de colores mucho más variada, mejorando de esta forma la resolución y el número de colores.\vspace{\baselineskip}

Otro gran avance en este año fue la creación del software ``Renderman'', el cual es un software de renderizado con el que se lograron realizar algunas películas animadas que han marcado la historia, e incluso el inicio de la misma con la película ``Toy Story'' la cual se convierte en la primera película de la historia creada totalmente por ordenador. \cite{Renderman}  \vspace{\baselineskip}

Otra característica importante de Renderman es que Pixar liberó este software de manera gratuita para quien desee utilizarlo siempre y cuando su uso no sea comercial, lo que permite su uso en investigación y desarrollo de herramientas así como soporte para la educación.\vspace{\baselineskip} 

\subsection{(1992) - Establecimiento de las especificaciones de OpenGL }

Este año se caracterizó por el aporte de un grupo de empresas con la creación del ``OpenGL Architecture Review Board'' (OpenGL ARB) con la cual se extienden las especificaciones del OpenGL para ese año y todos los años posteriores declarando así sus características y principales estándares, como se logra apreciar en \cite{OpenGL} .\vspace{\baselineskip}

\subsection{(1993) - Creación del primer navegador web gráfico}

El 22 de abril de este año se crea Mosaic, siendo este el primer navegador gráfico y el cual marca el inicio o la raíz de otros proyectos como Internet Explorer o Netscape \cite{Mosaic}. Este navegador logró ser adapatado de forma exitosa por las personas gracias a su entorno gráfico y aunque inicialmente se desarrolló para Unix, posteriormente apareció en otras plataformas como Windows y Mac OS .\vspace{\baselineskip}

\subsection{(1995) - Primera películada generada a computadora y creación de las primeras tarjetas gráficas 2D/3D}

Este año marca todo un avance en la industria cinematográfica al estrenarse la película ``Toy story'' a nivel mundial, la cual sería el primer largometraje creado completamente a computadora. Esta película dirigida por John Lasseter y estrenada en noviembre de ese año y creada en conjunto entre Walt Disney Pictures y Pixar, se conviertió en el primer largometraje de pixar y la primera cinta animada completamente con efectos digitales en la historia del cine \cite{ToyStory} .\vspace{\baselineskip}

Además también en este año se fabrican las primeras tarjetas gráficas 2D/3D por parte de la empresa Matrox, Creative, S3 y ATI, entre estas tarjetas destaca la creación de la tarjeta gráfica ``Mystique'', además de que estas tarjetas cumplían con el estándar SGVA.\vspace{\baselineskip}

\subsection{(1999) - NVIDIA domina el mercado de las tarjetas gráficas}

Es en este último año de la década de los años 90 en el cual NVIDIA inventa la GPU (Unidad de procesamiento gráfico ) la cual representa el corazón de una tarjeta gráfica y se encarga de realizar todos los cálculos complejos para que podamos disfrutar de elementos como los juegos en nuestras pantallas \cite{NVIDIA} .\vspace{\baselineskip}

\newpage
\begin{thebibliography}{X}
\bibitem{Historia} \textsc{Arroyo, L.},
\textit{Historia de los gráficos por computadora}, Timetoast Recuperado de: https://www.timetoast.com/timelines/historia-de-las-graficos-por-computadora

\bibitem{VGA} \textsc{Benbibre, C.},
\textit{Definición de VGA}, DefiniciónABC, Julio 2010. Recuperado de: https://www.definicionabc.com/tecnologia/vga.php 

\bibitem{ToyStory} \textsc{Demano, L.},
\textit{1995: Se estrena Toy Story, el primer largometraje creado totalmente con imágenes generadas por ordenador}, sinc, Noviembre 2010. Recuperado de: https://www.agenciasinc.es/Visual/Ilustraciones/1995-Se-estrena-Toy-Story-el-primer-largometraje-creado-totalmente-con-imagenes-generadas-por-ordenador

\bibitem{Renderman} \textsc{García, E.},
\textit{Renderman: el programa gratuito con el que Pixar hace sus pelis}, Adslzone, Enero 2004. Recuperado de: https://www.adslzone.net/reportajes/software/programa-pixar-peliculas/

\bibitem{NVIDIA} \textsc{López, J.},
\textit{Seguro que sabes lo que es una tarjeta gráfica pero ¿qué es una GPU?}, Hard Zone, Junio 2021. Recuperado de: https://hardzone.es/reportajes/que-es/gpu-caracteristicas-especificaciones/

\bibitem{Mosaic} \textsc{Pastor, J.},
\textit{Cuando Mosaic dominaba el mundo (de los navegadores}, Xataca, Abril 2018. Recuperado de: https://www.xataka.com/historia-tecnologica/cuando-mosaic-dominaba-el-mundo-de-los-navegadores

\bibitem{OpenGL} \textsc{Ríos, Y.},
\textit{OpenGL: ¿Qué es y para qué sirve?}, Profesional Review, Noviembre 2019. Recuperado de: https://www.profesionalreview.com/2019/11/15/opengl/

\end{thebibliography}

\end{document}